\begin{table}[!h]

\caption{\label{tab:top-5}This table presents the details for the districts labelled in Figure \ref{fig:holland}.}
\centering
\fontsize{8}{10}\selectfont
\begin{tabular}[t]{>{}l>{}lccccccc}
\toprule
\multicolumn{2}{c}{\textbf{ }} & \multicolumn{3}{c}{\textbf{Average of Simulations}} & \multicolumn{3}{c}{\textbf{ML Estimate}} & \multicolumn{1}{c}{\textbf{ }} \\
\cmidrule(l{3pt}r{3pt}){3-5} \cmidrule(l{3pt}r{3pt}){6-8}
\textbf{City} & \textbf{District} & \textbf{\% Change\textsuperscript{a}} & \textbf{From\textsuperscript{b}} & \textbf{To\textsuperscript{c}} & \textbf{\% Change} & \textbf{From} & \textbf{To} & \textbf{Shrinkage\textsuperscript{d}}\\
\midrule
 & \textbf{Usme} & 9\% & 5.5 & 5.8 & 7\% & 5.4 & 5.8 & 16\%\\

 & \textbf{Cuidad Bolivar} & 8\% & 6.0 & 6.4 & 7\% & 5.9 & 6.3 & 15\%\\

 & \textbf{Bosa} & 8\% & 7.1 & 7.4 & 7\% & 6.9 & 7.4 & 15\%\\

 & \textbf{San Cristobal} & 7\% & 12.6 & 13.4 & 6\% & 12.5 & 13.3 & 14\%\\

\multirow{-5}{*}{\raggedright\arraybackslash \textbf{Bogota}} & \textbf{Santa Fe} & 6\% & 27.3 & 29.0 & 5\% & 26.6 & 28.0 & 11\%\\
\cmidrule{1-9}
 & \textbf{Villa Maria El Triunfo} & 284\% & 5.3 & 19.5 & 264\% & 4.7 & 17.1 & 7\%\\

 & \textbf{Villa El Salvador} & 231\% & 7.3 & 23.5 & 217\% & 6.8 & 21.4 & 6\%\\

 & \textbf{Ventanilla} & 192\% & 8.4 & 23.4 & 182\% & 8.2 & 23.0 & 5\%\\

 & \textbf{Lurin} & 176\% & 6.9 & 17.4 & 168\% & 6.4 & 17.1 & 5\%\\

\multirow{-5}{*}{\raggedright\arraybackslash \textbf{Lima}} & \textbf{Chacalacayo} & 167\% & 6.7 & 16.4 & 159\% & 6.2 & 16.1 & 5\%\\
\cmidrule{1-9}
 & \textbf{La Pintana} & 642\% & 1.4 & 4.0 & 301\% & 0.8 & 3.4 & 53\%\\

 & \textbf{Cerro Navia} & 545\% & 1.5 & 4.2 & 272\% & 1.0 & 3.6 & 50\%\\

 & \textbf{Lo Espejo} & 517\% & 1.4 & 4.3 & 263\% & 1.0 & 3.5 & 49\%\\

 & \textbf{Renca} & 451\% & 1.3 & 4.1 & 241\% & 1.0 & 3.4 & 47\%\\

\multirow{-5}{*}{\raggedright\arraybackslash \textbf{Santiago}} & \textbf{San Ramon} & 451\% & 1.2 & 4.0 & 241\% & 1.0 & 3.3 & 47\%\\
\bottomrule
\multicolumn{9}{l}{\textsuperscript{a} Quantity of interest; percent change in enforcement operations when the percent in the lower class drops by half.}\\
\multicolumn{9}{l}{\textsuperscript{b} Enforcement operations when the percent in the lower class equals its observed value.}\\
\multicolumn{9}{l}{\textsuperscript{c} Enforcement operations when the percent in the lower class equals half its observed value.}\\
\multicolumn{9}{l}{\textsuperscript{d} Shrinkage in the quantity of interest due to switching from the average of simulations to the ML estimator.}\\
\end{tabular}
\end{table}